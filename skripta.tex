\documentclass[
  slovene,
  a4paper
]{book}

\usepackage[T1]{fontenc}
\usepackage[utf8]{inputenc}
\usepackage{babel}

\usepackage{color}
\usepackage{graphicx}
\usepackage{fullpage}
\usepackage{amsthm}


\newtheorem*{df}{Definicija}

\graphicspath{ {./slike/} }


\begin{document}

\title{Operacijski sistemi 2012/2013}
\author{prof. dr. Borut Robič }
\date{šolsko leto 2012/2013}

\maketitle

\chapter{Uvod}

V tem poglavju si bomo na kratko ogledali, kaj je sploh operacijski sistem (OS), zgodovinski potek razvoja OS in na zadnje še nekatere vrste operacijskih sistemov.


\section{Kaj je operacijski sistem?}

V grobem bi lahko računalniški sistem razdelili na tri glavne komponente:

\begin{description}
  \item[uporabnike (users)] Ti želijo rešiti računske probleme, tipični predstavniki te skupine pa bi bili ljudje, drugi računalniki in krmiljeni stroji.
  \item[uporabniške programe (application programs)] Ti opisujejo, kako rešiti računski problem (urejevalniki, prevajalniki, sistemi za upravljanje z podatkovnimi bazami).
  \item[strojno opremo (hardware, resources)] Ta zagotavlja vse računske vire, ki opravijo dejansko delo (procesor, pomnilnikm, V/I naprave ...).
\end{description}

\begin{figure}[h!]
  \centering
  \input{slike/001-racunalniski-sistem.pdf_tex}
  \caption{Uporabniki (U) uporabljajo razne uporabniške programe (P), s katerimi izkoriščajo razne računske vire (V)}
  \label{001-racunalsniski-sistem}
\end{figure}

Usklajevanje teh komponent pa je naloga operacijskega sistema, ki stoji med strojno opremo in sistemskimi ter uporabniškimi programi (slika~\ref{002-usklajevanje}). Pod pojmom sistemski programi pa ponavadi štejemo:
\begin{itemize}
  \item zbirnik (assembler)
  \item nalagalnik (loader)
  \item povezovalnik (linker), knjižnjice (libraries)
\end{itemize}

\begin{figure}
  \centering
  \input{slike/002-usklajevanje.pdf_tex}
  \caption{Hierarhična predstavitev umestitve operacijskega sistema v računalniški sistem.}
  \label{002-usklajevanje}
\end{figure}


\subsection{Naloge v OS}

V grobem bi lahko naloge OS razdelili v naslednje kategorije:
\begin{itemize}
  \item dodeljuje računalniške vire programom (z upoštevanjem potreb in zmožnosti računalniškega sistema)
  \item rešuje konfliktne situacije, kadar prihaja do tekme za vire (pojavi se potreba po optimizaciji)
  \item optimizacija izkoriščenosti strojne opreme
  \item nadzira delo nekaterih računalniških virov
  \item olajšuje delo uporabnikov
\end{itemize}

OS bi torej lahko opisali kot \emph{program, ki deluje kot posrednik med uporabnikom in strojno opremo.} Pri tem poskuša:
\begin{enumerate}
  \item zagotoviti unčikovito rabo računalniških virov
  \item ustvariti prijazno okolje za uporabnika
\end{enumerate}


\section{Zgodovina}

V tem odseku si bomo na kratko ogledali zgodovinski razvoj računalniških oz. operacijskih sistemov in osnovne ideje, ki so se pojavile med razvojem in jih uporabljamo še danes.


\subsection{Zgodnji računalniški sistemi}

Ti računalniki so bili fizično veliki in so se uporabljali s pomočjo konzole. Operater takšnega računalnika je bil obenem programer. Postopek izvedbe programa lahko opišemo z naslednjmi koraki:
\begin{enumerate}
  \item napiši program
  \item ročno ga naloži v pomnilnik
  \item napiši začetni naslov v PC
  \item sproži izvajanje in opazuj
  \item ob primernem času prekini izvajanje
  \item odčitaj in izpiši registre (rezultat)
  \item v kolikor pride med izvajanjem do težav, jih odpravi (razhroščevanje)
\end{enumerate}

Iz samega opisa je razvidno, da je takšno rokovanje z računalnikom zamudno in mučno. In ker je mukotrpnost nekega opravila najboljša stimulacija za izboljšanje postopka, so se kmalu pojavile prve izboljšave:
\begin{itemize}
  \item nova strojna oprema (čitalci luknjanih kartic, tiskalniki)
  \item programska oprema (zbirnik, nalagalnik, matematične knjižnjice, povezovalnik, gonilnik (skrbi za strojno opremo), prevajalniki, urejevalniki)
\end{itemize}

Te izboljšave so poenostavile programiranje, toda zapletle uporabo računalnika. Zakaj se je to zgodilo bomo razjasnili ob naslednji definiciji.

\begin{df}
  Posel (job) so vsi postopki, ki so potrebni, da se izvrši neko opravilo v računalniku
\end{df}

Primer: posel lahko zahteva veliko demontaže, montaže, eksplicitnega nalaganja, zaganjanja naših progrmaov => porabi se veliko časa\\*

//TODO slikca\\*

setup time(čas priprave)
\begin{math}
  T_1 - T_2
\end{math}
je zelo velik.\\*
Med tem je bil procesor(računalnik) neizkoriščen, toda računalnik je bil drag.\\*

\textcircled{?} Kako povečati izkoriščenost računalnika?\\*

Dve izboljšavi:\\*

\textcircled{1} profesionalizacija operaterja
//TODO slikca\\*
\textcircled{+} izkoriščenost je boljša\\*
\textcircled{-} neiteraktivnost: v primeru napake operater posreduje programerju izpise registrov in pomnilnilških lokacij(dump)\\*

\textcircled{2} Uvedba paketne obdelave\\*
//TODO slikca\\*
\textcircled{+} ker se posli s podobnimi zahtevami obdelujejo eden za drugim, se skrajša čas priprave => nekateri postopki so zato odpadli(optimizacija)\\*

Toda: kljub \textcircled{1} in \textcircled{2} izkoriščenost računalniškega sistema še vedno premajhna!\\*
Razlogi: operater
\begin{itemize}
  \item opazuje izvajanje programa
  \item ugotiviti mora, kaj se je zgodilo
  \item zagotoviti izpis pomnilnika(dump)
  \item naloži in sporoči naslednji posel iz paketa
  \item sproži izvajanje
\end{itemize}

Ugotovitev: med enim in drugim poslom še vedno mine preveč časa, spet je zavora človek(operater).\\*

Sklep: operaterja je treba zamenjati z strojem, s računalnikom samim. Računalnik sam naj skrbi  za neprekinjeno, avtomatično izvajanje nizanje poslov.\\*

Kako? Uvedba \textbf{rezidentnega monitorja}.

\section{Rezidentni monitor}

Za avtomatično izvajanje poslov so razvili program imenovan rezidentni monitor.\\*
Naloga: ob koncu posla mora rezidentni monitor avtomatično izvesti naslednji posel v paketu.\\*

Kako? V načelu:
\begin{enumerate}
  \item Ob zagonu računalnika se mora rezidentni monitor naložiti v glavni pomnilnik in poskrbeti za izvedbo prvega posla v paketu.\\* Sam rezidentni monitor med tem ostane v glavnem pomnilniku(rezidenten).

  \item Ko se posel konča, residentni monitor takoj prevzame nadzor\\*
in poskrbi za izvedbo naslednjega posla v paketu.
\end{enumerate}
//TODO slikca\\*

Kako naj rezidenti monitor izvede posel?

//TODO slikca\\*

Uvedejo se kontrolne kartice z navodili(ukazi) rezidentnemu monitorju.\\*
Posledica: en del rezidentnega monitorja mora skrbeti  za razpoznavanje in interpretacijo kontrolnih kartic\\*

Prednosti:
\begin{itemize}
  \item boljša izkoriščenost procesorja
\end{itemize}

Slabosti:
\begin{itemize}
  \item ni interakcije med programerjem in računalnikom
  \item čas od oddaje programa do rezultata (turn around) je bil relativno velik
\end{itemize}

Razloga:
\begin{itemize}
  \item računska zahtevnost problema(ok, ne moremo izboljšati)
  \item od tega kdaj posel pride na vrsto za obdelavo(moti programerje, lastnikov pa ne)
\end{itemize}

Toda: procesor bi lahko bil še bolj izkoriščen(ugotovitev)\\*

Razlog: počasnost V/I naprav, npr: 
\begin{itemize}
  \item procesor sposoben narediti nekaj
    \begin{math}
    10^6
    \end{math}
    ukazov / sekundo

  \item čitalec kartic 10 kartic/sekundo

  \item V/I naprava 10 000 počasnejša od procorja

  \item pozneje so izboljšali čitalce, vendar so procesorji napredovali še hitreje, razkorak še večji
\end{itemize}

Kaj sedaj? Dve izboljšavi:
\begin{itemize}
  \item ločeni V/I(offline processing)
  \item spooling
\end{itemize}

LOČEN V/I\\*
//TODO slikca\\*

Ob pojavu magnetnega traku\\*
//TODO slikca\\*

Prednost: boljša izkoriščenost procesorja(zaradi prekrivanja aktivnosti, overlapping)\\*
Slabost: turnarround še vedno velik, ni interakcije\\*

SPOOLING(simultaneous peripheral operations on-line)\\*
Pojavili so se magnetni diski(medij z direktnim dostopom => glava se relativno hitro premakne nad druge podatke)\\*
Ideja: Uporabiti magnetni disk kot velik vmesnik za začasno hranjenje V/I podatkov.\\*
//TODO slikca\\*

\textcircled{+} Prekrivanje V/I aktivnosti in računskih aktivnosti\\*
\textcircled{-} Se vedno ni interakcije\\*

Paketna obdelava + Multiprogramiranje\\*
Spooling omogoči, da je na magnetnem disku hkrati več pripravljenih poslov.\\*
Disk nudi direkten dostop do vsakega od njih.\\*
Posledica: možno je naseldnjega, ki se bo naložil v glavni pomnilni izbrati\\*
VPRAŠANJE: Kateri posel naj se naslednji naloži v glavni pomnilnik?\\*
ODGOVOR: Nova naloga za OS: Razvrščanje poslov(job scheduler)\\*
VPRAŠANJE: Kako izkoristiti možnost, da se v glavnem pomnilniku naloži več pripravljenih poslov?\\*
ODGOVOR: Multiprogramiranje - tehnika izvajanja več poslov(nova naloga za OS)\\*

//TODO slikca\\*

Multiprogramiranje
\begin{enumerate}
  \item v glavnem pomnilniku je OS in nekaj pripravljenih programov
  \item OS izbere enega med njimi in ga zažene
  \item če se tekoči program ustavi(bodisi, ker je končal ali pa nekaj čaka) prevzame nadzor OS in izbere drug pripravljen posel in ga požene
\end{enumerate}

//TODO slikca\\*

Nove naloge za OS?\\*
\textcircled{?} Kateri posli na se naložijo v glavni pomnilnik?\\*
RAZVRŠČANJE POSLOV\\*
\textcircled{?} Kje in koliko pomnilnika dodeliti izbranemu poslu?\\*
UPRAVLJANJE S POMNILNIKOM\\*
\textcircled{?} Kateremu od pripravljenih programov v glavnem pomniliku naslednjemu dodeliti procesor(multi programiranje)?\\*
RAZVRŠČANJE NA PROCESORJU\\*
\textcircled{?} Kako zaščititi posle v glavnem pomnilniku enega pred drugim?\\*
ZAŠČITA\\*
\textcircled{+} boljša izkoriščenost procesorja\\*
\textcircled{-} ni interakcije => manjša storilnost programerjev\\*

Interaktivnost (za uporabnika/programerja)\\*
Programerjeve želje:
\begin{enumerate}
  \item ukazovati OS preko tipkovnice in prejemati odzive/rezultate na zaslon
  \item OS naj "takoj" izvrši uporabnikov ukaz, uporabnika takoj obvesti o rezultatu in naj čaka na njegov ukaz
  \item v pomoč naj bo interaktivna programska oprema: urejevalniki, debugerji, ...\\*

//TODO slikca\\*
Posledice za razvijalce/lastnika
  \item uvesti bo treba t.i. datotečni sistem, ki
  	\item hrani programe/podatke v datotekah,
  	\item ki bodo shranjene na hitrih zunanjih pomnilnikih
	\item in bodo organizirane v mape itd.(zaradi hitrega dostopa, ...)
  \item cena
  \item pogosto bodo V/I operacije s človekom
  \item človek je počasen
\end{enumerate}

\textcircled{?} Posledice želje po iteraktivnosti je višja cena računalnika, hkrati pa grozi tudi slaba izkoriščenost računalnika(zaradi človekove počasnosti)


\end{document}
